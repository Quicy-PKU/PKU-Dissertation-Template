% 中文摘要
\begin{cabstract}
    在\href{https://www.overleaf.com/latex/templates/2021-peking-university-master-thesis-template-iofu728-pkuthss/rwfvbkpzydpf}{北京大学硕士学位论文模板 (iofu728)} 的基础上进行了以下改进:
    \begin{itemize}
        \item 将中文模板更改为英文模板。
        \item 将参考文献编译器从 Biber 更改为 BibTeX,避免原先的 textcite 等。
        \item 增加学位论文答辩委员会名单、博士学位论文答辩委员会决议书、提交终版学位论文承诺书 (在模板中均由版权声明代替,需要替换你自己的文件,参考 ``afterdefense.tex'' 文件)。
        \item 使 enumitem 支持 (1)、(a)、(i) 等格式,避免原先的 arabic, roman 等。
        \item 增加定理类环境和证明环境,增加插入表格、算法、图片等代码。
    \end{itemize}

    \bigskip
    \bigskip

    有任何疑问、建议、反馈可以通过邮件或 iMessage 联系我: zqye@quicy.cn。
    
    \bigskip
    \bigskip
    
    其他说明:
    \begin{itemize}
        \item 打个广告,高效使用 LaTeX 可参考:

\url{https://quicy.notion.site/LaTeX-6be09d441a594bed84d59dba2b254034}
        \item TexLive 2021 以上的版本在替换带有二维码的版权声明或其他文件时,会报错,目前暂无解决方法。本地需使用 2020 及以下版本的 TexLive 编译,Overleaf 可以在左上角 Menu 中更改 TexLive 版本为 2020。
        \item TexLive 2018 版本可能需要注释 pkuthss.cls 文件中的 \\RequirePackage\{chngcntr\}。
    \end{itemize}
    
    
    


\end{cabstract}

% 英文摘要
\begin{eabstract}
English abstract.
\end{eabstract}